\section{Discussion}
\label{sec:discussion}

The results presented in the previous section confirm that the integration of Gaussian Process Regression fundamentally transforms the practicality of the differential-algebraic approach to parameter estimation. Here, we discuss the implications of these findings, the reasons for GPR's success, and the inherent limitations of the method.

\subsection{Why the GPR-Enhanced Method Succeeds}
The dramatic improvement in performance over the original AAA-based algebraic method can be attributed to the fundamental difference between interpolation and regression. Interpolation methods, by definition, are forced to honor the data points exactly. When data is noisy, an interpolant will necessarily weave through the scattered points, leading to a highly oscillatory function whose derivatives are wildly inaccurate.

GPR, on the other hand, operates under a different paradigm. It assumes that the data is generated by a smooth, underlying function corrupted by noise. Instead of fitting the noise, GPR uses Bayesian inference to find the most likely smooth function given the data. This is achieved through two key mechanisms:
\begin{enumerate}
    \item \textbf{A Principled Smoothness Prior:} The choice of a smooth kernel (like the RBF kernel) acts as a "regularizer," penalizing functions that are not smooth and guiding the model to find a simpler, more plausible underlying signal.
    \item \textbf{Explicit Noise Modeling:} GPR does not assume the data is perfect. It explicitly models a noise term, and, crucially, learns the magnitude of this noise directly from the data. This allows it to automatically determine how much to trust the data points, smoothing more aggressively when the data is noisy and fitting more closely when the data is clean.
\end{enumerate}
By treating derivative estimation as a regression problem rather than an interpolation problem, we directly address the core weakness of the algebraic method, enabling it to function effectively in realistic, noisy conditions.

\subsection{Practical Implications for Researchers}
The successful robustification of the algebraic method has significant practical implications for researchers and engineers who build and use ODE models:
\begin{itemize}
    \item \textbf{Elimination of Initial Guesses:} The most immediate benefit is the removal of the need for manual tuning of initial parameter guesses. This lowers the barrier to entry for complex models and makes the estimation process more systematic and reproducible.
    \item \textbf{Built-in Identifiability Analysis:} Optimization methods typically return a single point estimate, giving no information about other possible solutions. Our method, by virtue of its root-finding approach, can return all possible parameter sets consistent with the data. This provides invaluable insight into the structural identifiability of a model, a feature often overlooked in standard parameter fitting workflows.
    \item \textbf{A New Tool for Model Exploration:} The GPR-enhanced algebraic method can be used as a standalone estimation tool, especially in low-to-moderate noise regimes. Alternatively, it can serve as a powerful initialization method for traditional optimization-based approaches. By providing a high-quality, data-driven starting point (or multiple starting points in the case of local identifiability), it can significantly improve the efficiency and reliability of subsequent local optimization.
\end{itemize}
In essence, this work makes the powerful theoretical advantages of the algebraic approach available as a practical tool for everyday modeling work.

\subsection{Limitations and Future Work}
Despite its strong performance, our method is not without limitations.
\begin{itemize}
    \item \textbf{Computational Cost:} The primary bottleneck is the computational complexity of both GPR and the polynomial solver. Standard GPR scales as $O(N^3)$ with the number of data points $N$, which can become prohibitive for very large datasets. Similarly, the difficulty of solving a polynomial system grows rapidly with the number of parameters and the degree of the equations. Future work should explore the use of sparse GPR approximations to improve scaling for large data.
    \item \textbf{Data Requirements:} As a two-stage method, our approach requires data that is sufficiently dense to allow for a reasonable regression fit. It is likely less suitable for extremely sparse data, where simulation-based methods that can integrate between points may have an advantage.
    \item \textbf{Reliance on Solvers:} The final step of the method depends on the successful convergence of a numerical polynomial solver. While modern homotopy continuation methods are very robust, they are not infallible and can struggle with poorly conditioned or very large systems.
\end{itemize}
Future research will also focus on propagating the uncertainty estimates from the GPR model through the entire pipeline, which could provide confidence intervals on the final parameter estimates---a feature currently lacking in our framework.
